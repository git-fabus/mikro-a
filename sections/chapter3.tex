\chapter{Marktmacht, Strategie und Marktversagen}


\section{Monopole}

Derzeit sind Preise als gegeben angenommen gewesen, doch was passiert, wenn wir diese Annahme ignorieren.
Stellen wir uns vor, dass ein Produkt nur einen Anbieter hat.


\begin{definition}[Monopol]
	Ein Unternhemen, dass als einziger Anbieter auftritt, hat ein \defemph{Monopol} in diesem Segment des Marktes
	\defemph{Monopsonien} sind die einzigen Käufer dieses Produkts.
\end{definition}

Informell gibt es auch \emph{Quasi-Monopole}, die strikt genommen nicht der einzige Anbieter sind,
aber den Markt stark dominieren.

\subsection{Entstehung von Monopolen}

Monopole entstehen auf verschiedene Weise, natürlich gegeben entsteht ein Monopol, wenn ein Unternehmen Kontrolle über knappe Ressourcen oder Inputs hat.
Auch die Effizienz kann ein Grund sein, denn wenige Anbieter können allgemein in diesen Situation für ein besseres Angebot sorgen.
Diese Art von Monopol wird oft man Staat kontrolliert oder überwacht.
Der Staat kann auch Monopole durch Rechtsstrukturen schaffen, um eine universelle Versorgung zu gewährleisten etc.


Im Monopol kann der Monopolist den Preis wählen – die nachgefragte
Menge ergibt sich dann aus der Nachfragefunktion.
\begin{construction}[Gewinnmaximierungsproblem des Monopolisten]
	Der Monopolist maximiert
	\[
		\max_p D(p) \cdot p - c(D(p))
		,\]
	wobei $D(p)$ die Nachfragefunktion und $c(D(p))$ die Kosten in Abhängigkeit der nachgefragten Menge ist.
\end{construction}


Inverse Nachfragefunktion tbd


Analog kann man auch das Monopolistenproblem umformulieren.

\begin{construction}[Alternative Maximierungsproblem des Monopolisten]
	Wir maximieren die Mengenwahl
	\[
		\max_q q \cdot p(q) - c(q)
		,\]
	wobei wir $r(q) = q \cdot p(q)$ als Umsatzfunktion des Monopolisten setzen.
\end{construction}

\begin{definition}
	Die Ableitung der Umsatzfunktion nennt man \defemph{Grenzerlös}:
	\[
		\operatorname{MR}(q)  = r'(q)
		.\]
\end{definition}
