\chapter{Begriffe}


\begin{lemma}[Shephard's Lemma] \index{Shephards's Lemma}
	Sei \( E(p, u) \) die minimale Ausgabenfunktion zur Erreichung des Nutzenniveaus \( u \) bei Preisen \( p \). Dann gilt:
	\[
		\frac{\partial E(p, u)}{\partial p_i} = h_i(p, u)
	\]
	wobei \( h_i(p, u) \) die kompensierte Nachfragefunktion nach Gut \( i \) ist.
\end{lemma}
\begin{lemma}[Shephard's Lemma]
	Sei \( E(p, u) \) die minimale Ausgabenfunktion zur Erreichung des Nutzenniveaus \( u \) bei Preisen \( p \). Dann gilt:
	\[
		\frac{\partial E(p, u)}{\partial p_i} = h_i(p, u)
	\]
	wobei \( h_i(p, u) \) die kompensierte Nachfragefunktion nach Gut \( i \) ist.
\end{lemma}

\begin{definition}[Marshallsche Nachfragefunktion] \index{Marshallsche Nachfragefunktion}
	Die Marshallsche Nachfragefunktion gibt für gegebenes Einkommen \( m \) und Güterpreise \( p \) die Konsummengen an, die der Konsument wählt, um seinen Nutzen zu maximieren:
	\[
		x(p, m) = \arg\max_{x \in \mathbb{R}^n_+} \, u(x) \; \text{mit} \; \sum_{i=1}^n p_i x_i \leq m
	\]
\end{definition}

