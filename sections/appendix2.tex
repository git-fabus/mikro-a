\chapter{Begriffe}


\begin{lemma}[Shephard's Lemma] \index{Shephards's Lemma}
	Sei \( E(p, u) \) die minimale Ausgabenfunktion zur Erreichung des Nutzenniveaus \( u \) bei Preisen \( p \). Dann gilt:
	\[
		\frac{\partial E(p, u)}{\partial p_i} = h_i(p, u)
	\]
	wobei \( h_i(p, u) \) die kompensierte Nachfragefunktion nach Gut \( i \) ist.
\end{lemma}
\begin{lemma}[Shephard's Lemma]
	Sei \( E(p, u) \) die minimale Ausgabenfunktion zur Erreichung des Nutzenniveaus \( u \) bei Preisen \( p \). Dann gilt:
	\[
		\frac{\partial E(p, u)}{\partial p_i} = h_i(p, u)
	\]
	wobei \( h_i(p, u) \) die kompensierte Nachfragefunktion nach Gut \( i \) ist.
\end{lemma}

\begin{definition}[Marshallsche Nachfragefunktion] \index{Marshallsche Nachfragefunktion}
	Die Marshallsche Nachfragefunktion gibt für gegebenes Einkommen \( m \) und Güterpreise \( p \) die Konsummengen an, die der Konsument wählt, um seinen Nutzen zu maximieren:
	\[
		x(p, m) = \arg\max_{x \in \mathbb{R}^n_+} \, u(x) \; \text{mit} \; \sum_{i=1}^n p_i x_i \leq m
	\]
\end{definition}






\chapter{Nützliche Lösungswege}

\section{Grundmodell: partielles Marktgleichgewicht}

Gegeben seien:
\begin{itemize}
    \item Nachfragefunktion \( D(p) \)
    \item Angebotsfunktion \( S(p) \)
\end{itemize}

Das Marktgleichgewicht ist definiert als der Preis \( p^* \), bei dem gilt:
\[
D(p^*) = S(p^*)
\]

Die Gleichgewichtsmenge ergibt sich durch Einsetzen von \( p^* \) in eine der beiden Funktionen:
\[
Q^* = D(p^*) = S(p^*)
\]

\subsection*{Beispielhafte lineare Funktionen}
\[
D(p) = a - b p, \qquad S(p) = c + d p
\]

Gleichgewicht:
\[
a - b p^* = c + d p^*
\]
\[
\Rightarrow p^* = \frac{a - c}{b + d}
\]
\[
Q^* = D(p^*) = a - b p^*
\]

\section{Offener Markt mit Weltmarktpreis}

Der Preis \( p_w \) ist exogen gegeben.
Im Gleichgewicht:
\begin{itemize}
    \item Konsumentenpreis: \( p_w \)
    \item Produzentenpreis: \( p_w \)
    \item Importe:
    \[
    M = D(p_w) - S(p_w)
    \]
\end{itemize}

\section{Markt mit Zoll}

Ein Zoll in Höhe von \( z \) erhöht den Konsumentenpreis:
\[
p_C = p_w + z
\]
Produzenten erhalten weiterhin:
\[
p_P = p_w
\]
Importmenge:
\[
M_z = D(p_w + z) - S(p_w)
\]

Zolleinnahmen des Staates:
\[
T = z \times M_z
\]

\section{Markt mit Importquote}

Die Importe sind auf eine maximale Menge \( M_{\text{max}} \) begrenzt.

Der inländische Gleichgewichtspreis \( p_Q \) ergibt sich aus:
\[
D(p_Q) - S(p_Q) = M_{\text{max}}
\]
Lösung:
\[
p_Q = \text{der Preis, der obige Gleichung erfüllt.}
\]

Anschließend:
\begin{itemize}
    \item Konsumentenpreis: \( p_Q \)
    \item Produzentenpreis: \( p_Q \)
    \item Konsummenge: \( D(p_Q) \)
    \item Produktionsmenge: \( S(p_Q) \)
    \item Importe: \( M_{\text{max}} \)
\end{itemize}

\section{Berechnung der Wohlfahrtsgrößen (optional)}

\subsection*{Konsumentenrente}
\[
KR = \int_0^{D(p)} \left[P(Q) - p \right] \, dQ
\]

\subsection*{Produzentenrente}
\[
PR = \int_0^{S(p)} \left[p - C(Q) \right] \, dQ
\]

\subsection*{Zolleinnahmen}
\[
T = z \times M_z
\]

\subsection*{Deadweight Loss (Wohlfahrtsverlust)}
Wohlfahrtsverluste ergeben sich aus der Reduktion von Konsum und Handel im Vergleich zum Freihandel.

\section{Zusammenfassung der Rechenwege}

\begin{table}[h!]
\centering
\begin{tabular}{@{}lll@{}}
\toprule
Fall & Gleichung zur Preisfindung & Importmenge \\ \midrule
Autark       & \( D(p) = S(p) \)                            & 0                         \\
Freihandel   & exogener Preis \( p_w \)                     & \( D(p_w) - S(p_w) \)     \\
Zoll \( z \) & Konsumentenpreis \( p_C = p_w + z \)         & \( D(p_w + z) - S(p_w) \) \\
Quote        & \( D(p_Q) - S(p_Q) = M_{\text{max}} \)       & \( M_{\text{max}} \)      \\ \bottomrule
\end{tabular}
\caption{Formale Berechnungen des partiellen Gleichgewichts}
\end{table}

